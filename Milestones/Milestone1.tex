\documentclass{sigchi}

% Use this section to set the ACM copyright statement (e.g. for
% preprints).  Consult the conference website for the camera-ready
% copyright statement.

% Copyright
\CopyrightYear{2016}
%\setcopyright{acmcopyright}
\setcopyright{acmlicensed}
%\setcopyright{rightsretained}
%\setcopyright{usgov}
%\setcopyright{usgovmixed}
%\setcopyright{cagov}
%\setcopyright{cagovmixed}
% DOI
\doi{http://dx.doi.org/10.475/123_4}
% ISBN
\isbn{123-4567-24-567/08/06}
%Conference
\conferenceinfo{CHI'16,}{May 07--12, 2016, San Jose, CA, USA}
%Price
\acmPrice{\$15.00}

% Use this command to override the default ACM copyright statement
% (e.g. for preprints).  Consult the conference website for the
% camera-ready copyright statement.


%% HOW TO OVERRIDE THE DEFAULT COPYRIGHT STRIP --
%% Please note you need to make sure the copy for your specific
%% license is used here!
% \toappear{
% Permission to make digital or hard copies of all or part of this work
% for personal or classroom use is granted without fee provided that
% copies are not made or distributed for profit or commercial advantage
% and that copies bear this notice and the full citation on the first
% page. Copyrights for components of this work owned by others than ACM
% must be honored. Abstracting with credit is permitted. To copy
% otherwise, or republish, to post on servers or to redistribute to
% lists, requires prior specific permission and/or a fee. Request
% permissions from \href{mailto:Permissions@acm.org}{Permissions@acm.org}. \\
% \emph{CHI '16},  May 07--12, 2016, San Jose, CA, USA \\
% ACM xxx-x-xxxx-xxxx-x/xx/xx\ldots \$15.00 \\
% DOI: \url{http://dx.doi.org/xx.xxxx/xxxxxxx.xxxxxxx}
% }

% Arabic page numbers for submission.  Remove this line to eliminate
% page numbers for the camera ready copy
% \pagenumbering{arabic}

% Load basic packages
\usepackage{balance}       % to better equalize the last page
\usepackage{graphics}      % for EPS, load graphicx instead 
\usepackage[T1]{fontenc}   % for umlauts and other diaeresis
\usepackage{txfonts}
\usepackage{mathptmx}
\usepackage[pdflang={en-US},pdftex]{hyperref}
\usepackage{color}
\usepackage{booktabs}
\usepackage{textcomp}

% Some optional stuff you might like/need.
\usepackage{microtype}        % Improved Tracking and Kerning
% \usepackage[all]{hypcap}    % Fixes bug in hyperref caption linking
\usepackage{ccicons}          % Cite your images correctly!
% \usepackage[utf8]{inputenc} % for a UTF8 editor only
\usepackage{float}

% If you want to use todo notes, marginpars etc. during creation of
% your draft document, you have to enable the "chi_draft" option for
% the document class. To do this, change the very first line to:
% "\documentclass[chi_draft]{sigchi}". You can then place todo notes
% by using the "\todo{...}"  command. Make sure to disable the draft
% option again before submitting your final document.
\usepackage{todonotes}

% Paper metadata (use plain text, for PDF inclusion and later
% re-using, if desired).  Use \emtpyauthor when submitting for review
% so you remain anonymous.
\def\plaintitle{}
\def\plainauthor{First Author, Second Author, Third Author,
  Fourth Author, Fifth Author, Sixth Author}
\def\emptyauthor{}
\def\plainkeywords{Authors' choice; of terms; separated; by
  semicolons; include commas, within terms only; required.}
\def\plaingeneralterms{Documentation, Standardization}

% llt: Define a global style for URLs, rather that the default one
\makeatletter
\def\url@leostyle{%
  \@ifundefined{selectfont}{
    \def\UrlFont{\sf}
  }{
    \def\UrlFont{\small\bf\ttfamily}
  }}
\makeatother
\urlstyle{leo}

% To make various LaTeX processors do the right thing with page size.
\def\pprw{8.5in}
\def\pprh{11in}
\special{papersize=\pprw,\pprh}
\setlength{\paperwidth}{\pprw}
\setlength{\paperheight}{\pprh}
\setlength{\pdfpagewidth}{\pprw}
\setlength{\pdfpageheight}{\pprh}

% Make sure hyperref comes last of your loaded packages, to give it a
% fighting chance of not being over-written, since its job is to
% redefine many LaTeX commands.
\definecolor{linkColor}{RGB}{6,125,233}
\hypersetup{%
  pdftitle={\plaintitle},
% Use \plainauthor for final version.
%  pdfauthor={\plainauthor},
  pdfauthor={\emptyauthor},
  pdfkeywords={\plainkeywords},
  pdfdisplaydoctitle=true, % For Accessibility
  bookmarksnumbered,
  pdfstartview={FitH},
  colorlinks,
  citecolor=black,
  filecolor=black,
  linkcolor=black,
  urlcolor=linkColor,
  breaklinks=true,
  hypertexnames=false
}

% create a shortcut to typeset table headings
% \newcommand\tabhead[1]{\small\textbf{#1}}

% End of preamble. Here it comes the document.
\begin{document}

\title{Milestone 1\\ Discovering Unaligned Expectations in Universities and Industry for New Graduates in Computer Science and Software Engineering and Finding Possible Solutions}

\numberofauthors{2}
\author{%
  \alignauthor{Alan Franzoni\\
    \affaddr{Georgia Institute of Technology}\\
    \affaddr{Trieste, Italy}\\
    \email{alan.franzoni@gatech.edu}}\\
  \alignauthor{Hasti Ghabel\\
    \affaddr{Georgia Institute of Technology}\\
    \affaddr{Atlanta, GA}\\
    \email{hghabel1@gatech.edu}}\\
}

\maketitle

\section{About our research}
We focus on discovering whether there is a misalignment in university and industry expectations for new graduates. We are considering four groups including undergraduate students, post-graduate students, professors and teachers and university staff, and industry professionals. In our final project, we will collect the data on what would be some available solutions to reduce the skill gap, which is caused by expectation gap among these four groups.

\textit{\textbf{Our Research Questions}}\newline
We will direct two sets of questions in our final project. In the first set, our main question is: \textbf{does the perceived skill gap in fresh graduates exist because the academy is unable to provide a good training, or just because a) the academy is not even trying to do that kind of job, and b) the industry is taking that kind of job for granted, or c) the students think they should be getting something that the university has no intention to provide them with?} Here, we ask what could be the reasons that there exist a gap between students, professors, and industry professionals' expectations. We will look into these questions for both \textbf{undergraduate} and \textbf{post-graduate} level students that are recently graduated from school. We also ask the question that \textbf{how the students' degree can improve the chance of getting hired? Do the graduate-level studies help the students to gain adaptive skills in industry more quickly?}\newline
In the second set of our questions, we are asking \textbf{what would be the best solutions that bring the university and industry's objectives closer to each other?} 


\textit{\textbf{Our Hypothesis}}\newline
\textit{Hypothesis 1}: One of the reasons for the perceived skill gap is that all those that should - in an employer's view - care for learning some skills to be used at work, don't actually have that aim during their education phase. In other words, the misaligned expectations between industry and university causes the skill gap. Those skills have impact on job proficiency and possibilities to get hired. We think that the expectations among four groups differ. These four groups are composed of \textbf{undergraduate-level students, post-graduate-level students, educators and school staff, and industry professionals}. The main problem is not only that one side hasn’t enough resources or skills to achieve a certain goal, but, rather, that there’s a different vision or gaol on what should be done, and different and unaligned \textit{rewards} exist for different groups.  

\textit{Hypothesis 2}:  We think that the following hypothesis is that the graduate-level studies, on the contrary, can play an important role on reducing the skill gap between industry and university, and therefore, can be considered as one good resolution for that issue. More specifically, we think that \textbf{high-quality online graduate-level programs} such as Georgia Tech OMSCS (Online Master of Science in Computer Science) are quite aligned with industry expectations. These such programs also target many people from all over the world, who can become proficient in their job quickly as well as being active in an academic environment.

\section{Preview research methodologies}
During this milestone, we focused on three of our sample groups: "undergraduate students", "professors, teachers, and university staff", "industry professionals". We provided three sets of surveys; one for each group. The surveys were written in GoogleFoms. You can find the link to the survey for each group below:\\

\textbf{Student (undergraduate level) group}:\\ \textcolor{blue}{https://goo.gl/forms/G4s5dtDZL9chZ1zt1}\\

\textbf{Professors, Teachers, and University Staff group}: \\ \textcolor{blue}{https://goo.gl/forms/jg20dElqkg1oCNiI2}\\

\textbf{Industry Professionals group}:\\  \textcolor{blue}{https://goo.gl/forms/4UCemOlDdaScKz952}\\

We provided the survey links with our classmates on PeerSurvey website (http://peersurvey.cc.gatech.edu/). We used our classmates as our small sample for sharing the survey with them. In order to recruit as many participants from our classmates as possible, we created a thread in Piazza, where we described our research goals and asked our classmates to take our survey if they are interested. We asked them to submit the survey for one group category that matches the best with their current position. 

In our surveys, we provided one last open-ended questions for the participants to give us feedback on their thoughts regarding the quality and quantity of the survey questions; Is there any questions that could be asked that we missed adding it in our survey? Is there any misunderstanding in any of the questions that could be described better for the participants? Here our goal was how we can improve the survey questions before sending it to the other general participants. We will describe the feedback in the later sections in this paper.

\section{Demo of the Website}
In order to send our survey to the general audience, we planned to create a website, where we can describe about our goals and the research. During this milestone, we set-up the website and put the content there. Our website can be found under link: \textcolor{blue}{http://www.misalignedtech.com/}
We will also add our finalized survey link there. This section will be completed after the survey becomes ready. In the end, we are planning to share the result we find on the website for whoever interested to see the results. 

We created a short demo of our website. Pleas use the link below to view the demo of our website.\\
\textcolor{blue}{***** YOUTUBE OR ANY OTHER LINKE HERE ***}

\section{Results}
We were able to collect total of 54 responses from all three surveys. The number of response break down for each of the surveys is as: Student survey = 28, Professor, Teachers, and University Staff survey = 3, and Industry Professionals survey = 23.
We will go through the results for each of the categories separately.

\textit{\textbf{STUDENTS}}\newline
The age of most of the participants in this category was 27 year old. Around 92\% of the participants live in the US and 8\% from other countries, which split into France and China.  Above 80\% of the participants' major (at undergraduate-level) was in computer science or any similar technology related fields. Some other, around 17\% were coming from other fields. Among those, 64\% of participants' latest degree is BS, around 28\% have their masters already, and around 7\% have already PHD. We found that most of the students received their degree form the US, but we also had other countries including China, India, and France.

We looked into the students' goals of studying computer science. We received about 75\% of the responses toward their interest in this field (figure 1).

\begin{figure}
\centering
  \includegraphics[width=1.05\columnwidth]{figures/goals_s}
  \caption{The student responses to the survey question: What are the current goals of studying computer science in undergraduate level? }~\label{fig:figure1}
\end{figure}

In the next level, we we interested to see what are the ideal goals of studying computer science. About 64\% of the participants' ideal goal was to learn about computer science and its fundamentals. Interestingly, the second high selected answer, about 57\%, was to get into industry very quickly (figure 2).

\begin{figure}
\centering
  \includegraphics[width=1.05\columnwidth]{figures/ideal_goals_s}
  \caption{The student responses to the survey question: In your opinion, what are the ideal goals of studying computer science in undergraduate level?}~\label{fig:figure2}
\end{figure}

We looked at some factors that can play a role in finding a job for new graduates. Some of those factors are GPA, university the student graduated, or gaining the skills from a university compared to learn from any online available courses. We created a rating questions which had the response options of "very important", "important", "neutral", "not important", and "not at all". The result for each of the questions are shown in a table in figure 3. As it shows, many students think that all three factors play an important role in finding a job when they graduate from school.

\begin{figure}
\centering
  \includegraphics[width=1.05\columnwidth]{figures/important_notimportant_table_s}
  \caption{The student rated each questions and selected one of the response options from "very important", "important", "neutral", "not important", and "not at all". Each row shows the item that can play a role in finding a job for new graduates. The columns are the response options}~\label{fig:figure3}
\end{figure}

Many students (39\%) think that there would be a transition time between 1-3 months for a new graduate to land on a job. 17\% of the students mentioned that the undergraduate student already have a job in a computer science related field (figure 4).

\begin{figure}
\centering
  \includegraphics[width=1.05\columnwidth]{figures/transition_time_s}
  \caption{The student responses to the survey question: For an undergraduate-level student, what would be the transition time from graduation to start the first job? }~\label{fig:figure4}
\end{figure}

We also looked in the the time required for a new graduate to become proficient in their future or current job. Interesting, the response options get very similar percentages.  Each option of "3-6 months" and "6-12 months" received 25\% of the responses. About 28\% of students think that the new graduates become proficient in their job in 1-3 months and 21\% of students think that proficiency in job takes more than 1 year (figure 5).

\textcolor{red}{****described the reasons they provided}

\begin{figure}
\centering
  \includegraphics[width=1.05\columnwidth]{figures/time_proficiency_s}
  \caption{The student responses to the survey question: What would be the required time for a new graduate to become proficient in their current/future job? }~\label{fig:figure5}
\end{figure}


\textit{\textbf{INDUSTRY PROFESSIONALS}}\newline
We had the age range from 22 to 55 years old. participants with the age of 26 and 34 years old had the highest percentage of participants (13\%).
Same as student group, around 92\% of the participants live in the US and 8\% from other countries, which split into Colombia and India. Among the participants, 94\% of participants' latest degree is BS, around 6\% have their master's degree. The average size of the organization division is about 600 people. 

Per industry professionals (78\%), the current goal of universities on teaching computer science is to teach undergraduate students the theoretical topics in this field. The other main goal was selected to be teaching analytical thinking skills (16\%). Only 10\% of the industry professionals believe that the goal of universities for undergraduate studies is to prepare students for their future job. 
We also asked that what would be the \textbf{ideal} goal of undergraduate studies to become successful in future job. The industry professionals believe that the ideal university goal should be teaching structures of computer science (69\%) and preparing students to get into industry very quickly (65\%).

We asked the same questions regarding the importance of GPA, university the student graduates, and gaining skills from degree studies, from industry professionals. The result is shown in the table in figure 6.
 
\begin{figure}
\centering
  \includegraphics[width=1.05\columnwidth]{figures/important_notimportant_table_i}
  \caption{The industry professionals rated each questions and selected one of the response options from "very important", "important", "neutral", "not important", and "not at all". Each row shows the item that can play a role in finding a job for new graduates. The columns are the response options}~\label{fig:figure6}
\end{figure}

Most of industry professionals that there would be 1-3 months time transition for new graduates in computer science to find a new job (34\%). We received the same amount of responses (34\%) that the undergraduate students work in a company during their studies (figure 7).

\begin{figure}
\centering
  \includegraphics[width=1.05\columnwidth]{figures/transition_time_i}
  \caption{The industry professionals responses to the survey question: For an undergraduate-level student, what would be the transition time from graduation to start the first job? }~\label{fig:figure7}
\end{figure}

Interestingly, opposite of students' opinions, industry professionals think that it takes about 1 year for a new graduate to become proficient in their future/current job (35\%). Only 4\% thinks that it will take 1-3 months to become proficient in the job (figure 8).

\begin{figure}
\centering
  \includegraphics[width=1.05\columnwidth]{figures/time_proficiency_i}
  \caption{The industry professionals responses to the survey question: What would be the required time for a new graduate to become proficient in their current/future job? *correction on the figure: the option of "more than 1 year: was exchanged with "more than 12 months" option.}~\label{fig:figure8}
\end{figure}


\textcolor{red}{****described the reasons they provided}


\textit{\textbf{Professor, Teachers, University Staff}}\newline
We did not receive a good amount of responses for this category. Only 3 people attended in this category. We cannot provide the results confidently. But one interesting point is to mention the current goal and ideal goal of academic environment on teaching computer science to undergraduate students, from professors and teachers point of view. Their votes for current goal is to teach programming languages and the ideal goal should be working on research topics in computer science fields. This point is very interesting, since it is very different from industry professionals' and also students' opinions.

\section{Survey Feedback}
Most of the participants gave us good comments and they liked our short survey questionnaire. We also received some productive feedback, which we will consider them in our next step. We collected some of the feedbacks and listed them below:
\begin{itemize}
	\item "I think that the survey was fine, I think it may help to understand what kind of school the survey respondent to? That could affect the results."
	\item "I think this survey did a great job covering the topic. I am interested to see the results."
	\item "It wasn't always clear if I was supposed to be answering from *purely* my perspective disregarding societal expectations. For example when answering "in my opinion having a high GPA [answer]" -- I personally don't care at all about GPA, but I acknowledge that many many other people do and therefore that means it ought to be a consideration. In some of your later questions you asked "why did you answer the way you did", which I think would be good to have for all like scale questions."
	\item "I don't like open-ended questions because they take longer to complete on a mobile phone."
	\item "spellcheck and ensure the student can answer the questions truthfully, leave options like "Other" and keep each question to one single question."
	\item "I think demographic information would be pertinent as well. I don't think you can put a general blanket over everyone (i.e. men vs. women might have differing ideas)"
	\item "It was a great survey, no issues"
	\item "Looks comprehensive as it is"
\end{itemize}


\section{Progress Report}
During this milestone, we were able to create the first draft of our survey and share it with our classmates. We've got total of 54 participants for all the categories. We asked from ourself how good is our survey and how the questions can help us to move toward our hypothesis. During collecting and analyzing this data sample, we learnt about the problems of some of the questions types. 
We received many feedbacks from the participants, which helped us to improve our survey questions. We applied many of the feedbacks and planed a new set of questions for the general audience survey. In the new survey, we applied the qualitative section to find what would be the best solutions to reduce the skill gap in industry for new graduates.

Some of our other achievements were setting up our website and testing it to make sure it opens in different browsers. We also decided to use R programming for our data analysis.

 \section{Future Steps}
In our next step, we are planning to create one single survey and put one questions that the participants can select their category. We will make it multiple selection option, since one participant can be a student as well as an industry professional.

We will have two types of questions in our survey. One would be multiple choice questions and the second part is open-ended questions. The first part focuses on discovery the misaligned expectations between undergraduate students, post-graduate students, professors, teachers and university staff and industry professionals. The second part is a qualitative type of survey and will focus on finding the solutions to reduce the skill gap in industry for new graduates. We need to collect the responses to the open-ended questions as soon as we receive any data. The qualitative analysis will be a little time consuming and needs to review and analyze all the answers.

Our plan is to refresh our knowledge in R programming and data analysis. We will use R programming tool to analyze the data and then, collect the results.
  

\balance{}

\balance{}

% REFERENCES FORMAT
% References must be the same font size as other body text.
\bibliographystyle{SIGCHI-Reference-Format}
\bibliography{../bibliography.bib}

\end{document}

%%% Local Variables:
%%% mode: latex
%%% TeX-master: t
%%% End:
 