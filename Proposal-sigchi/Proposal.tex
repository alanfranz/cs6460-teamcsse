\documentclass{sigchi}

% Use this section to set the ACM copyright statement (e.g. for
% preprints).  Consult the conference website for the camera-ready
% copyright statement.

% Copyright
\CopyrightYear{2016}
%\setcopyright{acmcopyright}
\setcopyright{acmlicensed}
%\setcopyright{rightsretained}
%\setcopyright{usgov}
%\setcopyright{usgovmixed}
%\setcopyright{cagov}
%\setcopyright{cagovmixed}
% DOI
\doi{http://dx.doi.org/10.475/123_4}
% ISBN
\isbn{123-4567-24-567/08/06}
%Conference
\conferenceinfo{CHI'16,}{May 07--12, 2016, San Jose, CA, USA}
%Price
\acmPrice{\$15.00}

% Use this command to override the default ACM copyright statement
% (e.g. for preprints).  Consult the conference website for the
% camera-ready copyright statement.

%% HOW TO OVERRIDE THE DEFAULT COPYRIGHT STRIP --
%% Please note you need to make sure the copy for your specific
%% license is used here!
% \toappear{
% Permission to make digital or hard copies of all or part of this work
% for personal or classroom use is granted without fee provided that
% copies are not made or distributed for profit or commercial advantage
% and that copies bear this notice and the full citation on the first
% page. Copyrights for components of this work owned by others than ACM
% must be honored. Abstracting with credit is permitted. To copy
% otherwise, or republish, to post on servers or to redistribute to
% lists, requires prior specific permission and/or a fee. Request
% permissions from \href{mailto:Permissions@acm.org}{Permissions@acm.org}. \\
% \emph{CHI '16},  May 07--12, 2016, San Jose, CA, USA \\
% ACM xxx-x-xxxx-xxxx-x/xx/xx\ldots \$15.00 \\
% DOI: \url{http://dx.doi.org/xx.xxxx/xxxxxxx.xxxxxxx}
% }

% Arabic page numbers for submission.  Remove this line to eliminate
% page numbers for the camera ready copy
% \pagenumbering{arabic}

% Load basic packages
\usepackage{balance}       % to better equalize the last page
\usepackage{graphics}      % for EPS, load graphicx instead 
\usepackage[T1]{fontenc}   % for umlauts and other diaeresis
\usepackage{txfonts}
\usepackage{mathptmx}
\usepackage[pdflang={en-US},pdftex]{hyperref}
\usepackage{color}
\usepackage{booktabs}
\usepackage{textcomp}

% Some optional stuff you might like/need.
\usepackage{microtype}        % Improved Tracking and Kerning
% \usepackage[all]{hypcap}    % Fixes bug in hyperref caption linking
\usepackage{ccicons}          % Cite your images correctly!
% \usepackage[utf8]{inputenc} % for a UTF8 editor only

% If you want to use todo notes, marginpars etc. during creation of
% your draft document, you have to enable the "chi_draft" option for
% the document class. To do this, change the very first line to:
% "\documentclass[chi_draft]{sigchi}". You can then place todo notes
% by using the "\todo{...}"  command. Make sure to disable the draft
% option again before submitting your final document.
\usepackage{todonotes}

% Paper metadata (use plain text, for PDF inclusion and later
% re-using, if desired).  Use \emtpyauthor when submitting for review
% so you remain anonymous.
\def\plaintitle{}
\def\plainauthor{First Author, Second Author, Third Author,
  Fourth Author, Fifth Author, Sixth Author}
\def\emptyauthor{}
\def\plainkeywords{Authors' choice; of terms; separated; by
  semicolons; include commas, within terms only; required.}
\def\plaingeneralterms{Documentation, Standardization}

% llt: Define a global style for URLs, rather that the default one
\makeatletter
\def\url@leostyle{%
  \@ifundefined{selectfont}{
    \def\UrlFont{\sf}
  }{
    \def\UrlFont{\small\bf\ttfamily}
  }}
\makeatother
\urlstyle{leo}

% To make various LaTeX processors do the right thing with page size.
\def\pprw{8.5in}
\def\pprh{11in}
\special{papersize=\pprw,\pprh}
\setlength{\paperwidth}{\pprw}
\setlength{\paperheight}{\pprh}
\setlength{\pdfpagewidth}{\pprw}
\setlength{\pdfpageheight}{\pprh}

% Make sure hyperref comes last of your loaded packages, to give it a
% fighting chance of not being over-written, since its job is to
% redefine many LaTeX commands.
\definecolor{linkColor}{RGB}{6,125,233}
\hypersetup{%
  pdftitle={\plaintitle},
% Use \plainauthor for final version.
%  pdfauthor={\plainauthor},
  pdfauthor={\emptyauthor},
  pdfkeywords={\plainkeywords},
  pdfdisplaydoctitle=true, % For Accessibility
  bookmarksnumbered,
  pdfstartview={FitH},
  colorlinks,
  citecolor=black,
  filecolor=black,
  linkcolor=black,
  urlcolor=linkColor,
  breaklinks=true,
  hypertexnames=false
}

% create a shortcut to typeset table headings
% \newcommand\tabhead[1]{\small\textbf{#1}}

% End of preamble. Here it comes the document.
\begin{document}

\title{Proposal:\\ Unaligned Expectations in Universities and Industry for New Graduates in Computer Science and Software Engineering}

\numberofauthors{2}
\author{%
  \alignauthor{Alan Franzoni\\
    \affaddr{for Submission}\\
    \affaddr{Trieste, Italy}\\
    \email{alan.franzoni@gatech.edu}}\\
  \alignauthor{Hasti Ghabel\\
    \affaddr{for Submission}\\
    \affaddr{Atlanta, GA}\\
    \email{hghabel1@gatech.edu}}\\
}

\maketitle

\section{Introduction}
There is a widespread agreement that new graduates from computer science and software engineering \textbf{do not always possess required skills, abilities or knowledge when joining the tech industry}: a lot of entry-level jobs actually require three years of experience \cite{Chakrabarti2018}; Gaps between Engineering Education, and Practice (what an Engineer does in real life) do exist \cite{Sivanesan2017}; The software industry presents dissatisfaction in relation to the level of recently graduated professionals \cite{Portela2017}; there is considerable room for improvement in what is taught to software students [in relation with job relevance] \cite{Lethbridgea}; Many employers find that graduates and sandwich students come to them poorly prepared for the every day problems encountered at the workplace \cite{Dawson2000}.\newline

Some universities and programs even took steps to try and fix this problem in some specific classes by doing all kind of things: from purposely hindering and disrupting the software development processes \cite{Dawson2000}, to adapting and incorporating industry training strategies into a software engineering course \cite{Portela2017}, to creating and adapting a project-based software engineering course that led the students to face with current, real-world engineering problems \cite{Delgado2017}, and to highlight to students how relevant is having and developing critical soft skills to succeed in projects. \cite{Bastarrica2017}.\newline

At first, we thought that differences could come from different programs, so we explored the difference between Computer Science and Software Engineering programs, but those didn't prove really relevant; the official ACM/IEEE curricula \cite{Force2013} \cite{Ardis2015} are somewhat overlapping, and some studies trying to highlight differences in outcomes between CS and SE graduates were mostly inconclusive: a lot of core competencies are quite similar \cite{Meziane2004} \cite{Rasool2014}. And, those recently-updated curricula don't seem to incorporate lessons from the aforementioned efforts.\newline

The acknowledgment of this skill gap and the efforts to train new graduates for the industry go back as far as 1992 \cite{Dawson1992}. So, \textbf{if in a quarter of a century little to nothing changed, what is the real matter}?	

\section{The hypothesis}
So, we started thinking that maybe the problem is not one of implementation; it's not the university is unable to train graduates for the industry. We began thinking that, maybe, there is a \textbf{misalignment in incentives and expectations} for the various stakeholders: industry practitioners, university teachers, and students.\newline

 So, the phenomenal question is: what are the expectations and motivations for students, teachers in the field of computer science and software engineering? Following that, what do employers expect from new graduates in this field? Is there any gap between the expectation of these groups?\newline
 
 We have found some research on the topic: students usually do not have a personal vision for what they hope to do with a Computer Science degree, and there's a mismatch between what they are taught and what they had expected \cite{Hewner2011}; some initial work on understanding what makes a good software engineer was performed \cite{Li2015}, but just on a few small samples from a very specific audience.\newline
 
 So, we would like to research into \textbf{what the expectations and motivations are} for our groups; we think we'll find a substantial \textbf{unalignment} between different groups' aims, that would justify the root reason for the perceived skill gap. So, we think that, most probably, employers will say they'll expect new graduates to be able to perform most real-world tasks; possibly, some students will say the same; but we suspect that most teachers won't answer this way, and they'll just say that they want their students to learn "computational thinking" or be prepared for doing research.
 
 We think, as well, that employers still find some usefulness in new CS/SE graduates, or they wouldn't hire them, and possibly students would not enter such kind of classes; we may able to infer what's the actual usefulness of somebody's studies when doing their jobs. Would it be a good idea to remove some course which is currently taken in order to improve the immediate perception of readiness for the industry, or would that lead to issues with skills later on?
 
 By the way, we seek mostly a \textbf{descriptive} approach to the problem, so we won't cry if the results don't turn out as we expect; we'll have verified a doubt we had.
 
 \section{What we plan to do}
 
 We plan to \textbf{create a survey} where we ask questions to assess the thoughts of students, university teachers, and industry practitioners about:
 \begin{itemize}
 	\item What should be the expected skills of fresh graduates (job proficiency, research proficiency, software writing, computational thinking, etc)
 	\item What are the actual skills they encounter "in the wild"
 	\item Whether they think gaps exist and should have been filled by somebody else?

 \end{itemize} 
 
 The survey will actually be tweaked depending on the target; some questions will be required for some groups, but not for others.
 
 What we perceive is novel is that \textbf{we target all groups at the same time}, so we can put their answers in perspective, and that we will \textbf{aim for a larger group than previous studies}; we'd really like to reach 300+ answers to our survey.
 
 In order to achieve such target, we'll create a website where we describe what we are doing, and we'll provide links to the survey and an easy way to share the website itself; we hope that a bit of advertising/social media push will help us getting to the numbers we need.
 
 Then, we'll collect the data, graph and analyze it, as well as creating a final presentation to sum up what we'll find.
 
 Our research will be a mix of survey-based and qualitative research. Some questions will be multiple-choice once, but some will be open-answer questions that we'll summarize later on.
 
 The independent variables will be the target groups. The dependent variables will be the groups' expectations and motivations.
 
 Threats to validity, both internal and external: we hope to get a good sample from around the world, but there's always the risk we incur in some biases - e.g. if some universities or organizations get really involved with the research we may get too many results from those specific institutions. We may also get a skewed number of answers from a specific target group - we'll need to understand what a good proportion is between groups as well, because we doubt the world offers the same amount of professionals, teachers, and students.
 
 
   


\section{*** Proposal TODO Items ***} 

A feedback:
What is the underlying educational problem or approach you see as a problem?  What are the main problem(s) with this approach or problem?  How can I fix it?

1- What are the research background that shows the motivation and expectations for student/teachers/employers in cs/se program?

2- what are the problems to limit the gap for new graduates in finding a new job after school? Why there is a gap in the new graduate skills and employer expectations?

3- what are some methodology that can shorten this gap? We can compare ivy league schools and other ones to see the differences and rate of employment after graduation?


\section{old introduction} 

One of the essential elements of a good software is to have a good software engineer (Paul Luo Li et al., 2015). The question is what makes a great software engineer? (Paul Luo Li et al., 2015) All different groups are looking into this question: employers want to hire a good software engineer, universities want to train a good engineer and new graduates want to become great (Paul Luo Li et al., 2015). Paul Luo Li et al. mention some of the employer’s expectations for hiring software developers (Paul Luo Li et al., 2015). The research indicates that the expert engineers are more productive in terms of producing faster solutions, produce more amount of code in the same amount of time, and write code with much fewer bugs (Paul Luo Li et al., 2015).\\ 
Hewner and Guzdial investigate a game company on what are the employer expectations from new graduates (Hewner and Guzdial, 2010). They identify two of the essentials skills or expectations are high programming skills as well as people skills such as working in a team and collaborating with other people (Hewner and Guzdial, 2010). McConnell argue that software developers’ personality traits like intellectual honesty, curiosity and being humble about their intelligence are important skills in addition to technical skills (McConnell, 2004). Hewner describes the mismatch between a student’s expectations on skills they hope to learn and what they are taught in an introductory computer science class (Hewner, 2011). He notes that students come to the course with preconception about what they will learn in that computer science course (Hewner, 2011). The educators mention some of the students preconceptions as below (Hewner, 2011):\\
	- Students expect to learn “advanced features” in application softwares.\\
	- They expect to do IT work such as assembling computers from parts and configure routers.\\
	- They expect to learn only about programming and not the architecture and theory.\\

Teaching computer science is different from teaching other subjects (Guzdial, 2014, https://cacm.acm.org/blogs/blog-cacm/174930-what-it-takes-to-be-a-successful-high-school-computer-science-teacher/fulltext ) Good teachers should be able to read the code and help students to write code by hand off from computers (as well as at the computer) (Guzdel, 2014). On the other hand, the less successful teachers focus heavily on assessments and readings (Guzdel, 2014). \\
The technology and computer science industries are growing so fast (Ayofe and Ajetola, 2009). Therefore, the companies are looking for the graduates, who are able to use the latest technologies. However, the companies criticize the universities curriculum doesn’t meet the practical issues in industry (Ayofe and Ajetola, 2009). 



\balance{}

\balance{}

% REFERENCES FORMAT
% References must be the same font size as other body text.
\bibliographystyle{SIGCHI-Reference-Format}
\bibliography{../bibliography.bib}

\end{document}

%%% Local Variables:
%%% mode: latex
%%% TeX-master: t
%%% End:
 